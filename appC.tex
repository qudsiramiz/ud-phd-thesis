%
% This is the Appendix C file (appC.tex)
%
\appendix{Gaussian Processes Algorithm} \label{apdx:C}

    This appendix details the Gaussian Processes Regression followed in Section~\ref{sec:meth8}. We
    start with the \texttt{ros} dataset. As discussed in \Cref{sec:3pic} the dataset has a
    resolution of $2048^3$ cells with a box size of $\sim 42\,d_{\rm i}$. Since the resolution of
    reconstructed image is of the order of $d_{\rm i}$, we decided to downsample the dataset using
    \texttt{block\_reduce} in \texttt{skimage}. We lowered the resolution to 42 cells along $x$ and
    $y$-directions, and 128 cells along $z$-direction. In units of $d_{\rm i}$, this gives us
    resolution of $\sim 0.3\,d_{\rm i}$ in $z$-direction and $1\,d_{\rm i}$ along the other two. The
    reason for different resolutions is the fact that along $xy$-plane, the resolution is limited by
    the minimum separation between two spacecraft ($1\,d_{\rm i}$), whereas along the $z$-direction
    it is the sampling rate of our instrument which determines the resolution.
    
    Here we list typical values of some of the parameters associated with the simulation and solar
    wind at 1\,au:
    \begin{align*}
        d_i \rm{(1\,au)} & \sim 100 ~\rm{km} \\
        V_{sw} & \sim 500 ~\rm{km/sec} \\
        X_{sim} \rm{(box size)} & \sim 40\,d_{\rm i} \\
        & \sim 4\times 10^3\,\rm{km} \\
        d_{\rm spc} & \sim [1, 11]\,d_{\rm i} \\
        & \sim \rm{ [10^2, 10^3]\,km} \\
        f_{\rm min} & \sim V_{sw}/(2\times d_{spc, min}) \\
        & \sim 2.5\,\rm{Hz} \\
    \end{align*}
    At the assumed solar wind speed of $500\,\rm{km/s}$, it takes the spacecraft configuration
    roughly 10 seconds to cross the whole box. And because of Nyquist criteria, we must have a
    sampling rate faster than $2.5\,\rm{Hz}$. Because of restrictions provided by Nyquist frequency
    and the box size and because we wanted to have a sample at every plane along the $z$-direction,
    we chose a sampling frequency of $40\,\rm{Hz}$. This is easily achievable for modern day
    magnetometers \citep{Bale2016, Russell2016}.

    Once we have the sampling rate and the spacecraft configuration fixed, we virtually fly through
    the simulation box and collect data for each spacecraft. At the end of simulated flight we have
    $n_{\rm spc}$ (number of spacecraft) number of data points for each plane and thus $128 \times
    n_{\rm spc}$ total data points along with their positions. We then define a kernel and
    associated variables as: \\
    \texttt{pln} ~\texttt{=}~ \texttt{`xy'} \\
    \texttt{drn} ~\texttt{=}~ \texttt{`z'} \\
    \texttt{ck\_len} ~\texttt{=}~ \texttt{5} \\
    \texttt{mat\_len\_scl} ~\texttt{=}~ \texttt{[2,2,6]} \\
    \texttt{mat\_nu} ~\texttt{=}~ \texttt{5/2} \\
    \texttt{sigma\_0} ~\texttt{=}~ \texttt{0} \\
    \texttt{n\_restarts\_optimizer} ~\texttt{=}~ \texttt{20} \\
    \texttt{kernel} ~\texttt{=}~ \texttt{ CK(ck\_len, (1e-2, 1e2)) + CK(ck\_len, (1e-2, 1e2)) *} \\
    \indent \indent \texttt{Matern(length\_scale=mat\_len\_scl, nu=mat\_nu)} \\
    \\
    we then run the Gaussian Processes with the selected kernel:\\
    \texttt{gp} ~\texttt{=}~\texttt{GaussianProcessRegressor(kernel=kernel, } \\
    \indent \indent \texttt{n\_restarts\_optimizer=n\_restarts\_optimizer)} \\
    \\
    We can then get the model based on the data collected by the spacecraft:\\
    \texttt{gp.fit(X, y)} \\
    \\
    and then provide \texttt{gp.fit} with the coordinate of all the locations at which we want to
    find the value of the field:\\ 
    \texttt{y\_pred, mse\_pred = gp.predict(x1x2x3, return\_std=True)} \\
    \\
    and reshape the output get the final value of field at each point:\\
    \texttt{Zp = np.reshape(y\_pred,(indx\_max - indx\_min + 1,}\\
    \indent \texttt{indy\_max - indy\_min + 1, indz\_max - indz\_min + 1, 3))} \\
    \texttt{MSE = np.reshape(mse\_pred,(indx\_max - indx\_min + 1,} \\
    \indent \texttt{indy\_max - indy\_min + 1, indz\_max - indz\_min + 1))} \\
    %\end{flalign*}

    \texttt{Zp} is the output that we have plotted in \Crefrange{fig:gpr_cas_bx}{fig:gpr_bm}.