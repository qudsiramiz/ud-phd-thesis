%
% This is Chapter 3 file (chap3.tex)
%
\chapter{Non-linear Plasma Dynamics}\label{chap:chap3}

    \section{Introduction to Turbulence}\label{sec:intr3}

        In laminar flow, different layers of fluid move smoothly without much mixing between layers,
        and the characteristic quantities/parameters - like velocity, pressure, or density - vary
        smoothly in a predictable way. A system where these quantities fluctuate in a chaotic
        fashion is called turbulent and the phenomena is called turbulence\index{turbulence}. Because of the chaotic
        nature of the fluctuations, unlike the laminar flow, prediction of the exact state of the
        system is essentially impossible. This makes a system extremely complicated to study, so
        much so that Sir Horace Lamb once remarked \citep{Goldstein1969}:

        \begin{quote}
             I'm an old man now, and when I die and go to heaven there are two matters on which I
             hope for enlightenment. One is quantum electrodynamics, and the other is turbulent
             motion of fluids. And about the former I'm rather optimistic.
            %\footnote{To be honest as an Indian, I find the idea of a $19^{th}$ century British's
            %assumption that they would go to heaven rather optimistic.}
        \end{quote}

        Turbulence is extremely complicated and the fact that it is ubiquitous in nature (and in
        most man made processes involving fluids) makes it
        unavoidable.%\footnote{Not that we would have avoided it anyway.}.

        Whether a given fluid system will develop turbulence largely depends on its viscosity, which
        is the liquid equivalent of friction and is a measure of how easy it is to for the liquid to
        flow \citep{Chapman1916, Jeans1905} and its Reynolds number
        \citep{Reynolds1883,Reynolds1886,Matthaeus1980}. A small Reynolds number\index{Reynolds number} means that the
        system is laminar whereas a high value Reynolds number implies turbulent flow. For a neutral
        fluid it is defined as:
        \begin{align}
            R_{\rm e} = L \frac{u}{\nu} \label{eq:rnld}
        \end{align}
        where $L$ is the characteristic length of the system, $u$ is the mean flow velocity and
        $\nu$ is the kinematic viscosity. For a similar amount of force or external pressure, a
        highly viscous fluid or one with low $R_{\rm e}$ can maintain laminar flow for much longer
        duration than a fluid with low viscosity or high $R_{\rm e}$. Presence of viscosity in a
        fluid leads to interaction between different layers or scales and results in energy transfer
        from larger to smaller scales through eddies which eventually reaches the smallest scale and
        dissipate as heat \citep{Kolmogorov1941a, Kolmogorov1941} in a process called energy cascade
        in turbulence. In a weakly collisional and magnetized plasma, the presence of charged
        particle and magnetic field complicates the process. For a system like solar wind, the
        situation is further complicated because of the relatively similar size of the system and
        the mean free path\footnote{Mean free path is defined as the average distance travelled by
        particles between two successive collisions.}\citep[Appendix 2]{Echim2010} both of which are
        of the order of 1\,au \citep[Table 1]{Verscharen2019} and thus one cannot use the classic
        methodology developed by \citet{Enskog1917} and \citet{Chapman1918}.

        Turbulence cascade has far reaching consequences for both neutral fluids and plasmas. It
        provides a pathway for the dissipation or transfer of energy from large scales, where they
        can be introduced, to smaller scales. In the next section (\Cref{sec:inter3b}) we will look
        at some of the consequences of turbulence in space plasmas. We discuss only those which are
        relevant to this thesis. In \Cref{sec:nlts} we discuss the linear and non-linear time scales
        associated with their respective phenomena.

    \section{Consequences of Turbulence in Space Plasmas} \label{sec:inter3b}

        In-situ observations and theoretical interpretations have established the ubiquitous
        presence of turbulence in space plasmas \citep[and references
        therein]{Matthaeus2011,Matthaeus2021}. In this section we discuss three of the major
        consequences that arise because of turbulence in space plasmas. Though these three are not
        the only consequences of turbulence, these were selected because of their relevance to this
        thesis, as we will see in \Cref{chap:chap5,chap:chap6,chap:chap7}.

        \subsection{Heating of plasma} \label{sec:hop}

            In space plasmas, under the assumption that the magnetic field changes slowly (slower
            than the ion gyrotropic time scale), the magnetic moment ($\mu$) of the particle is
            conserved \citep{Baumjohann1996,Verscharen2019}. Thus, we can write:
            \begin{align}
                \frac{d \mu}{dt} & = 0 \label{eq:mu_0}
            \end{align}
            where, $\mu = m_{\rm p}\,w_{\rm \perp p}^2/(2B)$, $m_{\rm p}$ is the proton mass,
            $w_{\rm \perp p}^2$ is the perpendicular thermal velocity and $B$ is the magnitude of
            magnetic field. Writing \Cref{eq:mu_0} in terms of the proton-perpendicular temperature
            using \Cref{eq:temp}, we have:
            \begin{align}
                \frac{d}{dt} \left(\frac{k_{\rm B} T_{\rm \perp p}}{B}\right) & = 0 \label{eq:mu_1}
            \end{align}
            or:
            \begin{align}
                T_{\rm \perp p} & \propto B \label{eq:mu_2}
            \end{align}
            In a similar vein for the parallel direction, under the assumption of no dissipation, we
            have:
            \begin{align}
                \frac{d}{dt} \left(\frac{k_{\rm B} T_{\rm \parallel p} B^2}{n_{\rm p}^2}\right) & = 0 \label{eq:mu_3}
            \end{align}
            or:
            \begin{align}
                T_{\rm \parallel p} & \propto \left(\frac{n_{\rm p}}{B}\right)^2 \label{eq:mu_4}
            \end{align}
            These two conservation laws (\Cref{eq:mu_2,eq:mu_4}) for the \textit{double-adiabatic
            invariants} are also called \textit{Chew–Goldberger–Low or CGL invariants} (for a bit
            more detailed discussion and derivation of CGL invariants, see see \Cref{apdx:B}).

            In the inner heliosphere, the magnitude of magnetic field (B) varies with solar distance
            as $B \propto r^{-1.5}$ \citep{Hellinger2013,Hanneson2020}, and the proton density
            varies as $n_{\rm p} \propto r^{-1.9}$ \citep{Hellinger2013}. If the CGL invariants were
            actively being conserved, the radial dependence for the perpendicular and parallel
            temperatures would be:
            \begin{align}
                %\begin{split}
                    T_{\rm \perp p} & \propto r^{-1.5} \label{eq:tperp_trend}\\
                    T_{\rm \parallel p} & \propto r^{-0.8} \label{eq:tpar_trend}
                %\end{split}
            \end{align}
            However, in-situ observations in the inner as well as outer heliosphere show a much
            flatter curve than those predicted by \Cref{eq:tperp_trend,eq:tpar_trend}. Based on
            Helios 1 and Helios 2 data, \citet{Hellinger2013} reported the value of exponents to be
            $-0.58$ and $-0.59$ for perpendicular and parallel temperatures respectively and $-0.58$
            for the scalar temperature for $r \in [0.3, 1]\,\rm{au}$.

            Flatter than expected temperature curves imply the existence of some mechanism which
            continues to heat the solar wind beyond the corona in both the parallel and
            perpendicular directions. Indeed, several studies
            \citep{ColemanJr1968,Verma1995,SorrisoValvo2007,MacBride2008} predict around
            $1000\,\rm{kJ/kg/sec}$ is being added as internal energy to the plasma at 1\,au.
            %\citet{Bandyopadhyay2020a} observed almost 2 order of magnitude higher heating rate
            %closer to the sun at $r \approx 0.2\,\rm{au}$.
            The dissipation of this energy at least partially accounts for the flatter radial trend
            in solar-wind temperature than that predicted by the double adiabatic expansion
            assumption.

        \subsection{Anisotropy} \label{sec:aniso}
        
            In \Cref{sec:instab2} we discussed the fact that because of anisotropy\index{anisotropy}, VDFs have excess
            free energy that results in development of microkinetic instabilities, though we did not
            discuss the origin of such anisotropies. As we saw in \Cref{sec:hop}, turbulence results
            in the transfer of heat from larger to smaller scales. However, in presence of an
            external background magnetic field the rate at which transfer occurs is not identical in
            each direction. Because of an uneven transfer along the parallel and perpendicular
            direction relative to the average magnetic field (inhibition along the direction of
            magnetic field), there is an imbalance between the amount of heating in different
            directions, resulting in anisotropy \citep{Shebalin1983,Oughton1994}.

        \subsection{Intermittency\index{Intermittency}}\label{sec:intmt}

            The solar wind at 1\,au exhibits localized structures that have been studied since the
            pioneering work of \citet{Burlaga1968}, \citet{Hudson1970}, \citet{Tsurutani1979}, and
            more recently by \citet{Ness2001}, \citet{Neugebauer2006}, \citet{ErdHoS2008}. Several
            studies have found evidence that plasma turbulence generates these structures
            dynamically \citep{Matthaeus1986, Veltri1999, Osman2013}. The structures are
            inhomogeneous and highly intermittent \citep{Osman2011, Osman2013,Greco2008}.
            Intermittency or burstiness in measured properties of turbulence is typically associated
            with the dynamical formation of coherent structures in space. These arise as a direct
            consequence of discontinuities in the magnetic field
            \citep{Greco2008,Greco2009,Vasquez2007}.

            One method for identifying a discontinuity in a time series of magnetic-field (or any
            other field in general) data is Partial Variance of Increments (PVI) \citep{Greco2008}.
            PVI is a powerful and reliable tool for identifying and locating such regions and it is
            unbiased towards any special structure since it cares only about the discontinuities in
            the magnetic field. This also manifests as a shortcoming of the technique since one
            cannot use it to study different kinds of discontinuities like radial or tangential
            discontinuities. \citet{Greco2008} defines PVI\index{PVI} as:
            \begin{align}
                \mathcal{I}(t, \delta t) & = \frac{|\Delta \mathbf{B}(t, \delta t)|}{
                    \sqrt{\langle |\Delta \mathbf{B}(t, \delta t)|^2 \rangle}} \label{eq:pvi}
            \end{align}
            where, $\Delta \mathbf{B}(t, \delta t) = \mathbf{B}(t+\delta t) - \mathbf{B}(t)$, is the
            vector increment in magnetic field at any given time $t$ and a time lag of $\delta t$.
            $\langle ...  \rangle$ is the ensemble average over a period of time, and $\mathcal{I}$
            is the normalized PVI. For studying local structures induced by turbulence, $\delta t$
            is typically chosen to be, assuming the validity of Taylor's hypothesis
            \citep{Taylor1938} which was found to be valid for inner heliosphere
            \citep{Chasapis2021}, of the order of $d_{\rm i}$.

    \section{Linear and Non-linear Time Scales} \label{sec:nlts}

        Since turbulence is not the only process that governs the dynamics, we must compare its
        characteristic timescale with other with those of other relevant processes. As we saw in
        \Cref{sec:instab2}, linear instabilities grow at growth rates of $\gamma_{\max}$. Inverse of
        $\gamma_{\max}$ gives us a linear time scale\index{time scale!linear} associated with such microinstabilities.
        \begin{align}
            \tau_{\rm lin} & = \frac{2\,\pi}{\left(\gamma_{\max}/\Omega_{\rm cp}\right)} \label{eq:lt}
        \end{align}
        Here we have scaled time scale with the proton cyclotron frequency ($\Omega_{\rm{cp}}$) to
        get a dimensionless timescale. This gives us an idea of timescales required by such linear
        processes to affect the local plasma.

        In a similar vein, one can compute nonlinear frequency associated with turbulence at any
        position \textbf{r} for a lag length scale of $\ell$ as follows\footnote{Ideally, velocity
        and not the magnetic field should be used for computing $\omega_{\rm nl}$. However, neither
        of the spacecraft data we used has enough resolution for such computation. We thus fall back
        to using magnetic field under the assumption of Alfv\'enic fluctuations.}:
        \begin{align}
            \omega_{\rm nl} \sim \delta b_\ell/\ell \label{eq:omnl}
        \end{align}
        where $\delta b_\ell$ is the change in the longitudinal magnetic field:
        \begin{align}
            \delta b_{\ell} = \left \lvert\hat{\boldsymbol{\ell}}
            \mathbf{\cdot} \left[\mathbf{b} (\mathbf{r} + \boldsymbol{\ell}) - \mathbf{b}
            (\mathbf{r})\right]\right\lvert \label{eq:db}
        \end{align}
        where \textbf{b} is the total magnetic field expressed in local Alfv\'en speed units
        ($\mathbf{b} = \mathbf{B}/\sqrt{\mu_\circ n_{\rm p} m_{\rm p}}$). Thus nonlinear time scale\index{time scale!nonlinear}
        has the expression:
        \begin{align}
            \tau_{\rm nl} & = \frac{2\,\pi}{\left(\omega_{\rm nl}/\Omega_{\rm cp}\right)} \label{eq:nlt}
        \end{align}
        These two processes under certain conditions might compete with each other and depending on
        the value of other kinetic or turbulent parameters one or the other may dominate. A
        simplistic understanding of this competition would imply that if one time scale is
        significantly smaller than the other, then the processes associated with former time scale
        will dominate the dynamics of the plasma. However, as we will see in \Cref{chap:chap7} the
        situation is a bit more complicated than that.