%
% This is Chapter 9 file (chap9.tex)
%
\chapter{Conclusion}\label{chap:chap9}

    \section{Broader Context} \label{sec:bcc9}

        Space plasmas such as the solar wind and terrestrial magnetosheath are highly structured and
        dynamic systems. Over the last few decades, two different theoretical frameworks have been
        developed to study their formation and evolution. The first uses linear Vlasov theory to
        explore micro-scale phenomena: the effects of waves and the constraints imposed by
        microinstabilities on the plasmas. The second includes the larger mesoscales and focuses on
        non-linear processes such as turbulence and the coherent structures it generates.

        Though both these frameworks have strong observational supports (see \Cref{sec:app2}), they
        are incompatible as traditionally formulated. Linear theory explicitly assumes a homogeneous
        background for the linear fluctuations that it studies. In contrast, turbulence produces
        strong inhomogeneities at all scales --- including those of linear theory. Reconciling these
        incongruous theories has motivated the work of this thesis.

        We focused our work on ion temperature-anisotropy and heating. Many previous studies have
        identified the heating of solar wind and magnetosheath to be ubiquitous but strongly
        inhomogenous (see \Cref{sec:inter3b}). The rate of heating varies across time and space and
        is often highly anisotropic which leads to strong temperature-anisotropies
        ($T_{\perp}/T_{\parallel} \neq 1$). The observations and simulations strongly indicate that
        turbulence produces inhomogeneous and anisotropic heating (see \Cref{sec:res5,sec:diss6}).
        Likewise, the predicted constraints of linear Vlasov theory align well with the observed
        distribution of ion temperature-anisotropy (see \Cref{sec:app2}). Despite their
        contradictory assumptions, both turbulence and microinstabilities seem to substantially
        affect ion-temperature in space plasmas. We conjectured that though turbulence produces
        inhomogeneities, the plasma remains sufficiently homogeneous at the kinetic micro-scales for
        the fastest modes of linear instability to develop.

    \section{Summary of Key Results} \label{sec:summ9}

        In \Cref{chap:chap5} we reported our analysis on the interplay of temperature-anisotropy
        driven linear microkinetic instabilities and intermittency arising as a consequence of
        turbulence. We showed that the two processes occur in close physical space. We also found
        the indication that the linear instabilities occur in discrete regions or intervals in
        different kinds of simulations as well as in in-situ data from space plasmas (see
        \Crefrange{fig:brjhb}{fig:brjwnd}).

        In \Cref{chap:chap6} we studied how intermittent structures affect the heating of the
        nascent solar wind and the terrestrial magnetosheath.. We used PVI to quantify
        intermittency. Study \citep{Osman2012a} using the same technique at 1\,au shows similar
        result thereby suggesting the ubiquitous nature of PVI heating. While we observed strong
        positive correlation between PVI and radial proton-temperature for the solar wind,
        magnetosheath plasma show little correlation. As discussed in \Cref{sec:diss6} some of the
        reasons for poor correlation might be the lower average value of PVI in magnetosheath as
        compared to the solar wind, small duration of observations as well as the relatively longer
        duration of the PVI events might be other contributing factors. In any case, this remains
        poorly understood and rather surprising given the positive correlation observed between PVI
        and the electron temperature \citep{Chasapis2018}. For the solar wind, we also found
        elevated value of conditionally averaged radial temperature up to one correlation length
        away from the point of a PVI event (see \Cref{fig:tem_pvi_lag}).

        In \Cref{chap:chap7} we compare the characteristic time scales of microinstabilities to
        those of turbulence at the same size scales for 6 different datasets. We observed that for
        the vast majority of data points/regions where the conditions were unstable, turbulence time
        scale is shorter than linear time scale (see
        \Crefrange{fig:ratio_allsim}{fig:ratio_kde_all}). This means that linear instabilities
        rarely have enough time to grow and affect the plasma before turbulence changes the plasma
        conditions driving the instability. However, since anisotropy is well regulated by the
        instability thresholds, we looked at the relative values of two time scales along the edges
        of Brazil plot. We found that along the edges linear instabilities do become faster than
        their turbulence counterpart and thus are able to regulate the extreme values of anisotropy
        (see \Crefrange{fig:ratio_brz_mms}{fig:ratio_brz_wnd}).

        At present we cannot image full structure of the interplanetary magnetic field using a few
        spacecraft. However, constellations with larger and larger number of spacecraft are becoming
        increasingly common with several possible missions to be launched in near future. Thus with
        a view towards the future, we carried out a study in \Cref{chap:chap8} to reconstruct the
        3-D topology and morphology of the interplanetary magnetic field from observations made by
        such a constellation with finite number of spacecraft. Using Gaussian Processes in machine
        learning for different configurations of number and arrangement of spacecraft, we showed
        that we need a baseline of 24 spacecraft to successfully carry out such a process. A
        complete 3-D image of the magnetic field will significantly advance our understanding of
        turbulence in space plasmas and and shed light on the exact process of turbulence cascade.