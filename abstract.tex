%
% This is the abstract A file (abstract.tex)
%
%\Abstract{Title of Abstract}
Space plasmas in the inner heliosphere exist in a weakly collisional and turbulent state. Though
energy transfer from large scales to smaller scales by turbulent cascade is widely accepted as an
important feature of space plasmas, details of its exact dissipation process are lacking. Features
arising because of turbulence, such as intermittency and temperature anisotropy, play important
roles in the dynamics of space plasmas. Microkinetic linear instabilities induced by temperature
anisotropy have been shown to change the statistical characteristics of plasma in a significant way.

Since the two processes, turbulence cascade and microkinetic instabilities, occur in the same
physical and phase space, there is an interplay at work. In this study we investigated this
interplay and the subsequent competition arising between the two, linear and nonlinear, processes.
We found an explicit connection between intermittency and linear instability growth rates. We also
showed localization of temperature enhancement regions along the intermittent structures, which in
turn can trigger linear instabilities. Investigation of the two processes shed light on why linear
theory works as well as it does, and shows the complicated nature of their interplay.

Information related to the exact spatial structure of the interplanetary magnetic field is vital to
our understanding of the type of turbulence active in the space plasmas and the mechanism of
turbulence cascade. This will help us discern the interplay between the two processes. We thus also
report on a proof of concept study of magnetic field topology reconstruction using Gaussian
Processes in machine learning.