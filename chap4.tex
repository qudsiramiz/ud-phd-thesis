%
% This is Chapter 4 file (chap4.tex)
%
\chapter{Datasets and Analysis Methods}\label{chap:chap4}

    We utilized multiple datasets (both observational and theoretical) while working on this thesis.
    This chapter gives a brief overview of all the datasets. \Cref{tab:datachap} summarizes these
    datasets and how they were used.

    \section{Spacecraft datasets}\label{sec:intr41}

        \subsection{Wind}\label{sec:wind}

            The Wind spacecraft\index{spacecraft!Wind} was launched on November 1, 1994 as part of international Solar
            terrestrial Physics (ISTP) with the objective of studying plasma processes in the solar
            wind near earth and in magnetosphere and ionosphere \citep{Acuna1995}. Wind is spin
            stabilized and makes one complete rotation every $\sim$\,3 seconds about axis aligned to
            perpendicular to the ecliptic plane \citep{Acuna1995, WilsonIII2021}. Wind's instruments
            collectively produce $\sim$\,1,100 data variables or datasets \citep{WilsonIII2021}. The
            instruments of interest to this thesis are the Magnetic Field Investigation (MFI) and
            the Faraday Cup (FC). 

            \paragraph*{MFI}\label{sec:mfi}

                MFI consists of two fluxgate magnetometers mounted on a boom at distances of 8 and
                12 meters from the spacecraft \citep{Lepping1995}. Though occasionally MFI can
                provide data as fast as 44\,Sa/s\footnote{Samples per second.} with great accuracy
                ($< 0.08\,\rm nT$), though 10.9\,Sa/s is the standard product and was used for this
                thesis.

            \paragraph*{FC}\label{sec:fc}

                The Solar Wind Experiment (SWE) suite includes two Faraday cups (FC)
                \citep{Ogilvie1995}. Each cup measures the current from incoming charged ions for a
                different energy bin during each rotation measuring current in 20 different look
                directions. It has 31\,energy bins which defines its resolution of the VDF. Since
                each rotation lasts about 3\,seconds, it takes FC roughly 93\,seconds to collect the
                full spectra. The current can then be converted to velocity of particle assuming an
                appropriate charge to mass ratio. Since it takes roughly 93\,seconds to get the full
                VDF, we get one measurement of parameters like density, velocity, temperature etc.
                every 93\,seconds. Consequently, while pairing FC data with MFI data, we further
                averaged MFI data to 93\,Sa/s. For an in depth discussion of extracting VDF from FC
                observation and computation of higher order moments see \citet{Maruca2012a}.

            \paragraph*{The \texttt{wnd} dataset} \label{sec:wndds}

                In this thesis we use the Wind data from 1994 to 2008, which henceforth shall be
                referred as \texttt{wnd} dataset. In the initial data cleaning process we discarded
                any point which had $R_{\rm p} < 0.1$ or $R_{\rm p} > 10$. We also only selected
                data from the pristine solar wind and discarded everything within the bow shock
                region of the Earth. A more detailed description of the data selection process can
                be found in \citet[\S4.1]{Maruca2012a}.

                \textbf{Computing linear growth rate and non-linear frequency}: In order to compute
                the value of linear growth rates at any point, we use the methodology mentioned in
                \Cref{sec:cgr} by using the local values of $R_{\rm p}$ and $\beta_{\parallel \rm
                p}$. We computed $\omega_{\rm nl}$ using \Cref{eq:omnl}, where we used
                $x$-component\footnote{For Wind, $x$-direction is defined by the line joining the
                Earth and the Sun.} of magnetic field for the longitudinal direction of \textbf{B}.
                Use of \textbf{x}-component instead of radial component introduces a small error in
                the computation of $\omega_{\rm nl}$ since the magnetic field at 1\,au is not
                perfectly aligned with the radial direction (on average, the angle between magnetic
                field and radial direction is $45^\circ$). The field also strongly fluctuates around
                the average value. Alfv\'en speed was computed using the average field from MFI data
                and $n_{\rm p}$ from FC as per equation \Cref{eq:alfv}. For lag we used $\ell =
                1/k_{\max}$, where $k_{\max}$ is the wave number corresponding to $\gamma_{\max}$.
                The lag of was taken as $1/k_{\max}$ in order to ensure that both $\gamma_{\max}$
                and $\omega_{\rm nl}$ are being computed at the same scale.

        \subsection{MMS}\label{sec:mms}

            Magnetospheric Multiscale (MMS)\index{spacecraft!MMS} is a constellation of four spacecraft which was launched
            by NASA on March 12, 2015. Main objective of the mission was to study how reconnection
            happens in a collisionless plasma in the Earths magnetosphere \citep{Russell2016}. MMS
            has 6 major instrument suites \citep{Russell2016} and in this thesis we used the data
            from FIELDS and Fast Plasma Investigation (FPI).

            \paragraph*{FIELDS} \label{sec:fields}

                The FIELDS instrument suite consists of 2 different kind of fluxgate magnetometers,
                a search coil magnetometer and an electron drift instrument \citep{Torbert2016}. The
                flux gate magnetometers are mounted at the end of two 5\,m booms of each spacecraft
                \citep{Russell2016}. The cadence of this FGMs is 128\,Hz meaning we get 128 samples
                of magnetic field vector every 1 second with an accuracy of $\sim$\,0.1\,nT
                \citep{Russell2016, Torbert2016}.

            \paragraph*{FPI} \label{sec:fpi}

                FPI uses electrostatic analyzer to measure the VDF of ions and electrons
                \citep{Pollock2016}. It has $\mathrm{180}^\circ$ instantaneous polar field of view
                at a resolution of $\mathrm{15}^\circ$. We use the proton density and temperature
                anisotropy which are among the standard products of FPI. FPI works in 2 modes:

                (a) \textbf{Slow/Survey mode}: which gives full 3-D VDF of ions every 1\,second.

                (b) \textbf{Fast/Burst Mode}: which gives 1 measurement of ion VDF every 150\,ms.

                \paragraph*{\texttt{mms} dataset} \label{sec:mmsds}

                    Though in burst mode cadence of FPI is very high they generally last for only a
                    few minutes. In our studies we thus used data from several different burst modes
                    spread over multiple years and when the spacecraft was in magnetosheath.
                    \Cref{tab:mmsdata} lists out all the dates and time from which data was used as
                    well as gives value of the plasma parameters.
                    \begin{table}[ht]
                        \centering
                        \caption[\texttt{mms} dataset details]{Burst data duration and median values of some plasma parameters}
                        \begin{tabular}{  m{0.20\linewidth}  m{0.20\linewidth}  m{0.20\linewidth}
                        m{0.25\linewidth}  }
                            \hline
                            \\
                            \multirow{2}{\linewidth}{Date \scriptsize{(YYYYMMDD)}} &
                            \multicolumn{2}{|c|}{Time \scriptsize{(HH:MM:SS) (GMT)}} &
                            \multirow{2}{\linewidth}{Median Values} \\
                            \cline{2-3}
                            & Start \newline HH:MM:SS & End \newline HH:MM:SS & \\
                            \hline
                            \\
                            20160111 & 00:57:04 & 01:00:33 & $n_{\rm p}$ = 52.04 $cm^{-3}$, \newline
                            $v_{\rm p}$ = 261.47 $km/s$, \newline $T_{\rm p}$ = 2.53 $\times
                            \mathrm{10^6} K$, \newline $R_{\rm p}$ = 1.09, \newline
                            $\beta_{\parallel \rm p}$ = 6.54\\
                            \\
                            \hline
                            \\
                            20160124 & 23:36:14 & 23:47:33 & $n_{\rm p}$ = 32.57 $cm^{-3}$, \newline
                            $v_{\rm p}$ = 242.21 $km/s$, \newline $T_{\rm p}$ = 3.98 $\times
                            \mathrm{10^6} K$, \newline $R_{\rm p}$ = 0.99, \newline
                            $\beta_{\parallel \rm p}$ = 12.57\\ \\
                            \hline
                            \\
                            %20160125 & & & \\
                            %\hline
                            20170118 & 00:45:54 & 00:49:43 & $n_{\rm p}$ = 198.26 $cm^{-3}$,
                            \newline
                            $v_{\rm p}$ = 135.11 $km/s$, \newline $T_{\rm p}$ = 1.31 $\times
                            \mathrm{10^6} K$, \newline $R_{\rm p}$ = 0.97, \newline
                            $\beta_{\parallel \rm p}$ = 10.66\\ \\
                            \hline
                            \\
                            %20170127 & & & \\
                            %\hline
                            20171226 & 06:12:43 & 06:52:22 & $n_{\rm p}$ = 22.29 $cm^{-3}$, \newline
                            $v_{\rm p}$ = 243.50 $km/s$, \newline $T_{\rm p}$ = 2.66 $\times
                            \mathrm{10^6} K$, \newline $R_{\rm p}$ = 1.04, \newline
                            $\beta_{\parallel \rm p}$ = 4.29\\
                            \\
                            \hline
                            %20181103 & & & \\
                            %\hline
                            \\
                            All & & & $n_{\rm p}$ = 2.94 $cm^{-3}$, \newline $v_{\rm p}$ = 240.15
                            $km/s$, \newline $T_{\rm p}$ = 2.74 $\times \mathrm{10^6} K$, \newline
                            $R_{\rm p}$ = 1.01, \newline $\beta_{\parallel \rm p}$ = 5.34\\
                            \\
                            \hline
                        \end{tabular}
                        \label{tab:mmsdata}
                    \end{table}

            Once we have the required parameters we compute other derived parameters like $\gamma$
            and $\omega_{\rm {nl}}$ in the same way as mentioned in \Cref{sec:wndds}. We refer to
            the complete MMS dataset as \texttt{mms}.


        \subsection{PSP}\label{sec:psp}

            Parker Solar Probe\index{spacecraft!PSP} was launched on August 12, 2018 with the objective to understand the
            dynamical structure of the sun, study and find the processes behind coronal heating and
            find out the process that accelerates energetic particles \citep{Fox2015}. The
            spacecraft has 4 major instrument suites: FIELDS, SWEAP, WISPR, \isois \citep{Fox2015}.

            \paragraph*{FIELDS}\label{sec:fields2}

                With main objective of measuring wave and turbulence in the inner heliosphere FIELDS
                measures the magnetic field using both, search coils and fluxgate magnetometers
                \citep{Bale2016}. All three magnetometers are mounted on a boom (search coil at
                3.08\,m and 2 magnetometers at 1.9\,m and 2.7\,m). For this thesis we use the
                magnetic field data from flux gate magnetometer. At the highest cadence magnetometer
                can record field at a rate of 292.969\,Sa/s or 256\,Sa/NYS, where 1\,NYsecond is
                defined as 0.837\,seconds \citep{Bale2016}\footnote{An alternate and definitely more
                magically colorful definition of a New York second is given by Sir Terry Pratchett
                as ``The shortest unit of time in the multiverse is the New York Second, defined as
                the period of time between the traffic lights turning green and the cab behind you
                honking.”}. Though for this thesis we mostly used data recorded at a slightly lower
                cadence of 64\,Sa/S unless otherwise specified.

            \paragraph*{SWEAP}\label{sec:sweap}

                Solar Wind Electrons Alphas and Protons or SWEAP is the particle instrument suite on
                PSP and is comprised of 4 sensor instruments and provides complete measurement of
                electron, alpha and protons which makes up for almost 99\% of solar wind
                \citep{Kasper2016}. Solar Probe Cup (SPC) and Solar Probe Analyzer (SPAN) make up
                SWEAP. We are mostly interested in SPC which is a fast Faraday cup and looks
                directly at the sun to measure the ion flux and its angle. The native cadence of SPC
                is 1\,Hz or 1\,Sa/s at an angular resolution of $\mathrm{10}^\circ$, though in
                another mode cadence can go as high as 16\,Hz at $\mathrm{1}^\circ$ resolution
                \citep{Kasper2016}. For this thesis we used 1\,Hz data from SPC. Though for the
                purpose of computation of anisotropy we resampled the data to 0.1\,Hz (see
                \Cref{chap:chap7}).

            \paragraph*{\texttt{psp} dataset} \label{sec:pspds}

                We used the PSP data from its first encounter with the Sun (October 31 to November
                11, 2018). From SPC we got the radial proton temperature/thermal speed. Since SPC
                only measures radial temperature, and proton temperature is significantly
                anisotropic \citep{Huang2020}, for computation of $\beta_{\parallel \rm p}$ we
                needed to ensure that the temperature we were measuring was indeed parallel
                temperature. Thus, we only considered data points where magnetic field was mostly
                radial. Any interval where the angle between $B_{\rm r} \mathbf{\hat{r}}$ and
                $\mathbf{B}$ was more than 30\,degrees was not considered. This ensured that the
                temperature measured by SPC was indeed the parallel temperature. We compute
                temperature anisotropy at a much lower cadence than the temperature measurement
                ($\sim$\,0.1\,Sa/S) using the methodology described in \citet{Huang2020}. Once we
                have the anisotropy data along with proton density and magnetic field strength we
                compute the $\beta_{\parallel \rm p}$ according to \Cref{eq:beta}. We then calculate
                $\gamma$ and $\omega_{\rm {nl}}$ using the same methodology as mentioned in
                \Cref{sec:wndds}. 

    \section{Simulation datasets}\label{sec:intr42}

        Though spacecraft provide plenty of in-situ data, because of several restrictions (e.g.,
        cost, planning, resolution, and cadence) not every phenomena of plasma can be studied using
        spacecraft data. Thus physicists often use simulations to study different systems or verify
        predictions made by theories under certain conditions. For space plasmas there are 3 types
        of simulations that are usually carried out.

        \subsection{MHD Simulations}\label{sec:mhd}

            MHD simulation\index{simulation!MHD} treats the plasma as an electromagnetic, conducting fluid having one
            characteristic velocity and temperature and studies its dynamics by numerically solving
            the required MHD equations. For more details about the underlying physics and some of
            the relevant equations see \citep{Hossain1995}.

        \subsection{Hybrid Simulations}\label{sec:hybd}

            In hybrid simulations\index{simulation!hybrid}, instead of treating the whole system as a fluid, electrons are
            treated as massless fluid and protons are treated as massive particles. For the details
            of such simulations and equations used for it refer to \citep{Terasawa1986, Vasquez1995,
            Parashar2009}.

        \subsection{Kinetic Simulations}\label{sec:kntc}

            In kinetic simulations\index{simulation!kinetic} with particle in cell (PIC) we solve Vlasov\index{Vlasov} equation (see
            \Cref{eq:vlas}) along with Maxwell's equation (see \Crefrange{eq:maxwell1}{eq:maxwell4})
            by treating plasma as a collection of individual particles. PIC simulations are often
            performed on either a 2.5D system or a full 3-D system.

        \paragraph*{3-D PIC Simulations}\label{sec:3pic}

            In full 3-D system\index{simulation!3-D PIC} the parameters are setup such that the vectors can fluctuate in all
            three directions. For this thesis, we used the output of a fully kinetic 3-D simulation
            performed by \citep{Roytershteyn2015}. In the simulation the system was initially
            perturbed ($|\delta\mathbf{B}^2| = \mathbf{B}_{\rm 0}^2$) and was then left to evolve
            under its own forcing. The undisturbed state of particle distribution was Maxwellian
            (for both proton and electron) at equal temperature ($T_{\rm p} = T_{\rm e}$). Some
            other parameters were $\beta_{\rm p}=\beta_{\rm e}=\, 0.5, R_{\rm p}=1, \omega_{\rm
            pe}/\Omega_{\rm ce} = 2$, $m_{\rm p}/m_{\rm e} = 50$ and the background magnetic field
            was in $z$-direction. Size of the box was $l \approx 42\,d_{\rm p}$, with a resolution
            of $2048^3$ cells. Average number of particles in each cell was 150 making a total of
            $\sim\,2.6 \times 10^{12}$. We refer to this dataset as \texttt{ros}.

        \paragraph*{2.5-D Simulations}\label{sec:2pic}

            In case of a 2.5D simulation\index{simulation!2.5-D} the plasma parameters are allowed to vary only in 2
            dimensions, though they have all 3 components. Depending on the direction of background
            magnetic field one can further classify 2.5-D simulation in following classes:

            (a) \textbf{2.5D perpendicular PIC simulation}: The parameters are allowed to vary only
            in 2 spatial dimensions with background magnetic field perpendicular to the simulation
            plane.

            (b) \textbf{2.5D oblique PIC simulation}: The parameters are allowed to vary only in 2
            spatial dimensions with background magnetic field neither parallel nor perpendicular to
            the simulation plane.

            (c) \textbf{2.5D parallel PIC simulation}: The parameters are allowed to vary only in 2
            spatial dimensions with background magnetic field parallel to the simulation plane.\\

        In this thesis we used both perpendicular and parallel simulations. For the 2.5-D
        perpendicular simulation we used the output from a P3D code \citep{Zeiler2002}. The initial
        conditions were such that we have $m_{\rm p}/m_{\rm e} = 25, T_{\rm p} = T_{\rm e},
        \beta_{\rm p} = \beta_{\rm e} = 0.6, \delta B = 0.5\,B_{\rm 0}$ and the length of the box
        was $l = 149.6\,d_{\rm p}$ at a resolution $4096^2$ of with each cell having an average of
        3200 particles with each species resulting in a total of $1.07 \times 10^{11}$ particles.
        For more details on the simulation refer to \citep{Parashar2018}. We refer to this dataset
        as \texttt{149p6}.

        We also used a 2.5-D parallel simulation where the background magnetic field was in the
        plane with $\mathbf{B}_{\rm 0} = B_{\rm 0} \hat{x}, m_{\rm p}/m_{\rm e} = 25, \omega_{\rm
        pe}/\Omega_{\rm ce} = 8, \beta_{\rm p}= \beta_{\rm e} = 0.6$. The size of the box was
        $l_\parallel = 149.6\,d_{\rm p}$ (in parallel direction) and $l_\perp = 37.4\,d_{\rm p}$ (in
        perpendicular direction) at a resolution of $4043 \times 1000$ with an average of 800
        particles/cell resulting in a total of $6.5 \times 10^{9}$ particles.  More information
        about this simulation can be found in \citep{Parashar2019, Gary2020}. We refer to this
        dataset as \texttt{kaw}.

        For 2 datasets of simulations (\texttt{kaw} and \texttt{149p6}), once we have the value of
        $R_{\rm p}$ and $\beta_{\parallel \rm p}$ we compute $\gamma$ and $\omega_{\rm nl}$ in the
        same way as mentioned in \Cref{sec:wndds}. For the case \texttt{ros}, for computation of
        $\omega_{\rm nl}$, because of some computational limitations, the value of lag was kept
        fixed at $1\,d_{\rm p}$.

        \begin{table}[ht]
            \centering
            \caption{Datasets used in this study}
            \begin{tabular}{ m{0.1\linewidth}  m{0.3\linewidth}  m{0.20\linewidth} m{0.3\linewidth}}
            \\
                \hline
                \\
                Dataset & Type of data & median values & List of chapters\\
                \\
                \hline
                \\
                \texttt{149p6} & PIC Simulation (2.5-D) & $R_{\rm p} = 0.89$, \newline
                $\beta_{\rm \parallel p} = 0.67$ & \Cref{chap:chap5,chap:chap7}\\ \\
                %\hline
                \\
                \texttt{kaw} & PIC Simulation (2.5-D) & $R_{\rm p} = 0.83$, \newline
                $\beta_{\parallel \rm p} = 0.64$ & \Cref{chap:chap5,chap:chap7} \\ \\
                %\hline
                \\
                \texttt{ros} & PIC Simulation (3-D) & $R_{\rm p} = 1.04 $, \newline
                $\beta_{\parallel \rm p} = 0.84$ & \Cref{chap:chap5,chap:chap7,chap:chap8} \\ \\
                %\hline
                \\
                \texttt{mms} &Spacecraft Observation (Magnetosheath) & see \Cref{tab:mmsdata} &
                \Cref{chap:chap5,chap:chap7} \\ \\
                %\hline
                \\
                \texttt{wnd} & Spacecraft Observation (Solar Wind at 1\,au) & $R_{\rm p} = 0.50$,
                \newline
                $\beta_{\parallel \rm p} = 0.69$ & \Cref{chap:chap5,chap:chap7} \\ \\
                %\hline
                \\
                \texttt{psp} & Spacecraft Observation (Solar Wind at 0.2\,au) & $R_{\rm p} = 1.44 $,
                \newline $\beta_{\parallel \rm p} = 0.50$ & \Cref{chap:chap6,chap:chap7} \\ \\
                \hline
            \end{tabular}
            \label{tab:datachap}
        \end{table}